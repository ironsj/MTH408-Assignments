\documentclass[11 pt]{article}
\title{Problem Set 3}
\usepackage{latexsym}
\usepackage{amssymb}
\usepackage{amsfonts}
\usepackage{amsmath}
\usepackage{amsthm}
\newtheorem{proposition}{Proposition}

\newcommand{\newpar}{\vspace{.15in}\noindent}

\begin{document}

\noindent Jake Irons, MTH 408-01, Problem Set 3, Problem 2

\noindent ironsj@mail.gvsu.edu
\newpar
\begin{proposition}
$\displaystyle{\lim_{x \to 0}}\frac{x^2}{\sqrt{x+1}-1}\sin(\frac{1}{x})=0$.
\end{proposition}
\begin{proof}
We claim $\displaystyle{\lim_{x \to 0}}\frac{x^2}{\sqrt{x+1}-1}\sin(\frac{1}{x})=0$. Let $f(x)=\frac{x^2}{\sqrt{x+1}-1}\sin(\frac{1}{x})$ where $x\neq0$. Since $\lvert\sin(\frac{1}{x})\rvert\le1$ for all $x\neq0$, $\lvert f(x)\rvert\le\lvert\frac{x^2}{\sqrt{x+1}-1}\rvert$ for all $x\neq0$. Therefore, $-\lvert\frac{x^2}{\sqrt{x+1}-1}\rvert\le f(x)\le\lvert \frac{x^2}{\sqrt{x+1}-1}\rvert$.

\newpar
We will now find $\displaystyle{\lim_{x \to 0}}\frac{x^2}{\sqrt{x+1}-1}$. If we multiply $\frac{x^2}{\sqrt{x+1}-1}$ by $\frac{\sqrt{x+1}+1}{\sqrt{x+1}+1}$, we get $\frac{x^2\sqrt{x+1}+x^2}{x}$. If we simplify, the result is $x\sqrt{x+1}+x$. If we pull out an $x$ the result is $x(\sqrt{x+1}+1)$. Therefore, for all $x\in\mathbb{R}$, $\displaystyle{\lim_{x \to 0}}x(\sqrt{x+1}+1)=0(\sqrt{0+1}+1)=0$. Since $\displaystyle{\lim_{x \to 0}}x(\sqrt{x+1}+1)=\displaystyle{\lim_{x \to 0}}\frac{x^2}{\sqrt{x+1}-1}$, $\displaystyle{\lim_{x \to 0}}\frac{x^2}{\sqrt{x+1}-1}=0$. Since $\displaystyle{\lim_{x \to 0}}\frac{x^2}{\sqrt{x+1}-1}=0$, $\displaystyle{\lim_{x \to 0}}\lvert \frac{x^2}{\sqrt{x+1}-1}\rvert=0$.

\newpar
Since $\displaystyle{\lim_{x \to 0}}\lvert \frac{x^2}{\sqrt{x+1}-1}\rvert=0$, by the Squeeze Theorem it follows that $\displaystyle{\lim_{x \to 0}}f(x)=0$. Therefore, $\displaystyle{\lim_{x \to 0}}\frac{x^2}{\sqrt{x+1}-1}\sin(\frac{1}{x})=0$.



\end{proof}

\newpar
\begin{proposition}
$\displaystyle{\lim_{x \to 0}\frac{\lfloor\frac{1}{x}\rfloor}{\frac{1}{x}}}=1$.
\end{proposition}
\begin{proof}
By the definition of a floor function $\lfloor x\rfloor$, $x-1<\lfloor x\rfloor\le x$ for each $x\in\mathbb{R}$. Therefore, for the floor function $\lfloor \frac{1}{x}\rfloor$ where $x\neq0$, $\frac{1}{x}-1<\lfloor\frac{1}{x}\rfloor\le\frac{1}{x}$. If we multiply this by $\frac{1}{\frac{1}{x}}$ where $x\neq0$, we have $\frac{\frac{1}{x}-1}{\frac{1}{x}}<\frac{\lfloor\frac{1}{x}\rfloor}{\frac{1}{x}}\le\frac{\frac{1}{x}}{\frac{1}{x}}$ for $x>0$ or $\frac{\frac{1}{x}-1}{\frac{1}{x}}>\frac{\lfloor\frac{1}{x}\rfloor}{\frac{1}{x}}\ge\frac{\frac{1}{x}}{\frac{1}{x}}$ for $x<0$.

\newpar
First we will consider $\displaystyle{\lim_{x \to 0}\frac{\frac{1}{x}-1}{\frac{1}{x}}}$. If we multiply $\frac{\frac{1}{x}-1}{\frac{1}{x}}$ by $\frac{x}{x}$ where $x\in\mathbb{R}$ and $x\neq0$, the result is $1-x$ where $x\in\mathbb{R}$ and $x\neq0$. Since $\frac{\frac{1}{x}-1}{\frac{1}{x}}=1-x$ where $x\in\mathbb{R}$ and $x\neq0$, we have $\displaystyle{\lim_{x \to 0}\frac{\frac{1}{x}-1}{\frac{1}{x}}}=\displaystyle{\lim_{x \to 0}}1-x$. If we take the limit of $1-x$ as $x$ approaches 0, the result is 1. Thus, $\displaystyle{\lim_{x \to 0}\frac{\frac{1}{x}-1}{\frac{1}{x}}}=1$.

\newpar
Now we consider $\displaystyle{\lim_{x \to 0}\frac{\frac{1}{x}}{\frac{1}{x}}}$. If we multiply $\frac{\frac{1}{x}}{\frac{1}{x}}$ by $\frac{x}{x}$ where $x\in\mathbb{R}$ and $x\neq0$, the result is 1. Since $\frac{\frac{1}{x}}{\frac{1}{x}}=1$ and $\displaystyle{\lim_{x \to 0}1}=1$, we have $\displaystyle{\lim_{x \to 0}\frac{\frac{1}{x}}{\frac{1}{x}}}=1$. 

\newpar
Since $\frac{\frac{1}{x}-1}{\frac{1}{x}}<\frac{\lfloor\frac{1}{x}\rfloor}{\frac{1}{x}}\le\frac{\frac{1}{x}}{\frac{1}{x}}$ or $\frac{\frac{1}{x}-1}{\frac{1}{x}}>\frac{\lfloor\frac{1}{x}\rfloor}{\frac{1}{x}}\ge\frac{\frac{1}{x}}{\frac{1}{x}}$ for any $x\in\mathbb{R}$ where $x\neq0$ and $\displaystyle{\lim_{x \to 0}\frac{\frac{1}{x}-1}{\frac{1}{x}}}=\displaystyle{\lim_{x \to 0}\frac{\frac{1}{x}}{\frac{1}{x}}}=1$, by the Squeeze Theorem, $\displaystyle{\lim_{x \to 0}\frac{\lfloor\frac{1}{x}\rfloor}{\frac{1}{x}}}=1$.

\end{proof}

\newpar
\begin{proposition}
$\displaystyle{\lim_{x \to 1}\lfloor x\rfloor x}$ does not exist.
\end{proposition}
\begin{proof}
Let $f(x)=\lfloor x\rfloor x$. We will use sequential argument to show that  $\displaystyle{\lim_{x \to 1}}f(x)$ does not exist. Consider two sequences $(x_n)$ and $(y_n)$ defined by $x_n=1-\frac{1}{n}$ and $y_n=1+\frac{1}{n}$, where $n\in\mathbb{N}$. Since $\frac{1}{n}$ will never be 0 for every $n\in\mathbb{N}$ both $x_n$ and $y_n$ will not equal 1. Since $\lim(1)=1$ and $\lim(\frac{1}{n})=0$, $\lim(1+\frac{1}{n})$ and $\lim(1-\frac{1}{n})$ both equal 1. Thus, $\lim(x_n)=\lim(y_n)=1$. Since $\lfloor x\rfloor=0$ when $0\le x<1$ and $\lfloor x\rfloor=1$ when $1\le x<2$, $f(x_n)=0\cdot x_n=0$ and $f(y_n)=1\cdot y_n=y_n$ for all $n\in\mathbb{N}$. Therefore, $\lim(f(x_n))=0$ and $\lim(f(y_n))=\lim(y_n)=1$. Hence $\lim(f(x_n))\neq\lim(f(y_n))$. By sequential criterion it follows that $\displaystyle{\lim_{x \to 1}\lfloor x\rfloor x}$ does not exist.
\end{proof}

\newpar
\begin{proposition}
$\displaystyle{\lim_{x \to 0}}f(x)=0$ where $f(x)=x^2$ if $x\in\mathbb{Q}$ and $f(x)=0$ if $x\not\in\mathbb{Q}$.
\end{proposition}
\begin{proof}
We know that if $f(x)$ is continuous at $x=0$, $\displaystyle{\lim_{x \to 0}}f(x)=f(0)$. Therefore, we will first show $f(x)$ is continuous at $x=0$. If $x=0$, then $f(0)=0^2=0$.

\newpar
We will now show that $g$ is continuous at $x=0$. For all $\epsilon>0$, if there exists $\delta>0$ such that $\lvert x\rvert<\delta$, then $\lvert f(x)-f(0)\rvert<\epsilon$. We know $f(0)=0$, so $\lvert f(x)-f(0)\rvert=\lvert f(x)\rvert$. We will first consider when $x\in\mathbb{Q}$. If $\lvert f(x)-f(0)\rvert<\epsilon$, then since $f(x)=x^2$ and $f(0)=0$, $\lvert f(x)-f(0)\rvert=\lvert x^2\rvert$. Thus, we have $\lvert x^2\rvert<\epsilon$.

\newpar
Let $\epsilon>0$ be given. We choose $\sqrt{\epsilon}=\delta$. Now if $\lvert x-0\rvert<\delta$, then $\lvert x^2\rvert<\delta^2$. Since $\delta=\sqrt{\epsilon}$, we have $\lvert x^2\rvert<\delta^2=\epsilon$. Since $\lvert f(x)-f(0)\rvert=\lvert x^2\rvert$, it follows that $\lvert f(x)-f(0)\rvert<\epsilon$.

\newpar
If $x$ is irrational, we have $\lvert f(x)-f(0)\rvert=0$. Since $\epsilon>0$, we have $\lvert f(x)-f(0)\rvert<\epsilon$. Thus, by the definition of continuous, $f$ is continuous at $x=0$.

\newpar
Since $f(x)$ is continuous at $x=0$, $\displaystyle{\lim_{x \to 0}}f(x)=f(0)$. Because $f(0)=0^2=0$, it follows that $\displaystyle{\lim_{x \to 0}}f(x)=0$.


\end{proof}
\newpage
\noindent Jake Irons, MTH 408-01, Problem Set 3, Problem 3

\noindent ironsj@mail.gvsu.edu
\newpar
\begin{proposition}
When $f$ is real valued function defined on $\mathbb{R}$ satisfying $\lvert f(x)\rvert\le\lvert x\rvert$ for all $x\in\mathbb{R}$, the value of $f(0)=0$.

\end{proposition}
\begin{proof}
We are given that $\lvert f(x)\rvert\le\lvert x\rvert$ for all $x\in\mathbb{R}$. If we put $x=0$, we have $\lvert f(0)\rvert\le\lvert 0\rvert$. Thus, $\lvert f(0)\rvert\le0$. Since $\lvert f(0)\rvert$ can not be less than 0, we have $f(0)=0$, completing the proof. 

\end{proof}

\newpar
\begin{proposition}
When $f$ is real valued function defined on $\mathbb{R}$ satisfying $\lvert f(x)\rvert\le\lvert x\rvert$ for all $x\in\mathbb{R}$, $f$ is continuous at $x=0$.

\end{proposition}
\begin{proof}
Suppose we want to show that for every $\epsilon>0$ there exists a $\delta>0$ such that $\mid f(x)-f(0)\mid<\epsilon$ for all $x$ with $\mid x-0\mid<\delta$. 

\newpar
Now, let $\epsilon>0$ be given. Choose $\delta=\epsilon$. We have previously shown $f(0)=0$, hence $\lvert f(x)-f(0)\rvert=\lvert f(x)-0\rvert$. Thus, we have $\lvert f(x)\rvert$, and we know from the proposition $\lvert f(x)\rvert\le\lvert x\rvert$. Since $\epsilon=\delta$, we have $\lvert f(x)\rvert\le\lvert x\rvert<\epsilon$. Since $\lvert f(x)\rvert<\epsilon$ and $f(0)=0$, $\lvert f(x)-f(0)\rvert<\epsilon$ for $\lvert x-0\rvert<\delta$. Thus, by Definition C.1.1, $f$ is continuous at $x=0$, completing the proof.
\end{proof}

\begin{proposition}
When $f$ is real valued function defined on $\mathbb{R}$ satisfying $\lvert f(x)\rvert\le\lvert x\rvert$ for all $x\in\mathbb{R}$, there exists a function $f$ which is not continuous at any $x\neq0$.

\end{proposition}
\begin{proof}
We define $f$ as a function where $f(x)=x$ if $x\in\mathbb{Q}$ and $f(x)=0$ if $x\not\in\mathbb{Q}$. We must first show that this function meets the requirements in the proposition, and then show it is not continuous at any $x\neq0$. If we let $a$ be any real number. Therefore, $a$ is either rational or irrational. If $a$ is rational, then $f(a)=a$. Thus, this satisfies that $\lvert f(x)\rvert$ must be less than or equal to $\lvert x \rvert$. If $a$ is irrational, then $f(a)=0$. Since $a$ is irrational, $\lvert a\rvert>0$. Thus, this also satisfies that $\lvert f(x)\rvert$ must be less than or equal to $\lvert x \rvert$. Therefore, the function we defined meets the requirements of $f$.


\newpar
Now we must show that our function is not continuous at any $x\neq0$. We will show that this function is only continuous at $x=0$. Let $a$ be any real number not equal to 0. Then $a$ is rational or irrational. If $a$ is rational, $f(a)=a$. By the Density Property, for each $n\in\mathbb{N}$, there exists $x_n\in(a,a+\frac{1}{n})$ such that $x_n\not\in\mathbb{Q}$. Thus, $a<x_n<a+\frac{1}{n}$ for all $n\in\mathbb{N}$. Since $\lim(\frac{1}{n})=0$ for all $n\in\mathbb{N}$ and $\lim(a)=a$, $\lim(a+\frac{1}{n})=a$. Thus, by the Squeeze Theorem, $\lim(x_n)=a$. But, $f(x_n)=0$ for all $n\in\mathbb{N}$ since $x_n\not\in\mathbb{Q}$. Thus, $\lim(f(x_n))=0\neq a$ since $a\neq0$. Therefore, by discontinuity criterion, $f$ is not continuous when $a\in\mathbb{Q}$.

\newpar
We now consider when $a\not\in\mathbb{Q}$ and $a\neq 0$. If $a\not\in\mathbb{Q}$, then $f(a)=0$. By the Density Property, for each $n\in\mathbb{N}$, there exists $y_n\in(a,a+\frac{1}{n})$ such that $y_n\in\mathbb{Q}$. Thus, $a<y_n<a+\frac{1}{n}$ for all $n\in\mathbb{N}$. Since $\lim(\frac{1}{n})=0$ for all $n\in\mathbb{N}$ and $\lim(a)=a$, $\lim(a+\frac{1}{n})=a$. Thus, by the Squeeze Theorem, $\lim(y_n)=a$. But, $f(y_n)=y_n$ for all $n\in\mathbb{N}$ since $y_n\in\mathbb{Q}$. Thus, $\lim(f(y_n))\neq0$. Therefore, by discontinuity criterion, $f$ is not continuous when $a\not\in\mathbb{Q}$. Since $f$ is discontinuous when $a\in\mathbb{Q}$ and $a\not\in\mathbb{Q}$, $f$ is not continuous when $a\neq0$.

\newpar
Now consider if $a=0$. Suppose we want to show that for every $\epsilon>0$ there exists a $\delta>0$ such that $\mid f(x)-f(0)\mid<\epsilon$ for all $x$ with $\mid x-0\mid<\delta$. 

\newpar
Now, let $\epsilon>0$ be given. Choose $\delta=\epsilon$. We have previously shown $f(0)=0$, hence $\lvert f(x)-f(0)\rvert=\lvert f(x)-0\rvert$. Thus, we have $\lvert f(x)\rvert$, and we know from the proposition $\lvert f(x)\rvert\le\lvert x\rvert$. Since $\epsilon=\delta$, we have $\lvert f(x)\rvert\le\lvert x\rvert<\epsilon$. Since $\lvert f(x)\rvert<\epsilon$ and $f(0)=0$, $\lvert f(x)-f(0)\rvert<\epsilon$ for $\lvert x-0\rvert<\delta$. Thus, by Definition C.1.1, $f$ is continuous at $x=0$.

\newpar
We have shown that $f$ is not continuous at any $x\neq 0$, completing the proof. 

\end{proof}
\newpage
\noindent Jake Irons, MTH 408-01, Problem Set 3, Problem 4

\noindent ironsj@mail.gvsu.edu
\newpar
\begin{proposition}
Suppose that $f$ is continuous on $[0, 1]$, $f(x)$ is irrational for every $x\in[0, 1]$, and $f(0)=\sqrt{2}$. Find $f(\frac{1}{2})$.
\end{proposition}
\begin{proof}
Let $f$ be continuous on $[0, 1]$, $f(x)$ be irrational for every $x\in[0, 1]$, and $f(0)=\sqrt{2}$. We claim $f(\frac{1}{2})=\sqrt{2}$. 

\newpar
Since $f$ is continuous on [0,1], the Intermediate Value Theorem tells us that $f$ will take any value between $f(0)$ and $f(1)$ at some point in the interval [0,1]. Since $f(0)=\sqrt{2}$, it will take every value between $\sqrt{2}$ and $f(1)$. 

\newpar
If $\sqrt{2}\neq f(1)$, we have $\sqrt{2}<f(1)$. Thus, the Density Property tells us there exists rational and irrational values between $\sqrt{2}$ and $f(1)$. Since $f(x)$ is irrational for every $x$ in [0,1], it would be impossible for $\sqrt{2}\neq f(1)$. Therefore, $\sqrt{2}=f(1)$ and $f$ has to be a constant function. Therefore, $f(\frac{1}{2})=\sqrt{2}$, completing the proof.


\end{proof}

\newpar
\begin{proposition}
If $f(x)=\sin(x+1)\cos(\frac{1}{x+1})-\ln(x+2)\cos(\frac{1}{x+1})$ if $x\neq-1$ and $f(x)=\ln(3k+2)$ if $x=-1$, then the value of $k$ so that $f$ is continuous at $x=-1$ is $-\frac{1}{3}$. 
\end{proposition}
\begin{proof}
We claim $k$ must equal $-\frac{1}{3}$ so that $f$ is continuous at $x=-1$. We know that $f(x)$ is continuous at $x=-1$ if $\displaystyle{\lim_{x \to -1}}f(x)=f(-1)=\ln(3k+2)$. We will first find $\displaystyle{\lim_{x \to -1}}f(x)$.

\newpar
If we pull $\cos(\frac{1}{x+1})$ out of  $\sin(x+1)\cos(\frac{1}{x+1})-\ln(x+2)\cos(\frac{1}{x+1})$, we get $(\cos(\frac{1}{x+1}))(\sin(x+1)-\ln(x+2))$. Since $0\le\lvert x\cos(x)\rvert\le\lvert x\rvert$, we have $0\le\lvert (\cos(\frac{1}{x+1}))(\sin(x+1)-\ln(x+2))\rvert\le\lvert \sin(x+1)-\ln(x+2)\rvert$ for all $x\neq-1$. By Theorem R.2.1, $\lvert (\cos(\frac{1}{x+1}))(\sin(x+1)-\ln(x+2))\rvert=\lvert (\cos(\frac{1}{x+1}))\rvert\lvert (\sin(x+1)-\ln(x+2))\rvert$. Thus,$0\le\lvert (\cos(\frac{1}{x+1}))\rvert\lvert (\sin(x+1)-\ln(x+2))\rvert\le\lvert \sin(x+1)-\ln(x+2)\rvert$ for all $x\neq-1$.  Because the limit of the sum of two functions is the sum of the limit of two functions, $\displaystyle{\lim_{x \to -1}}(\sin(x+1)-\ln(x+2))=\displaystyle{\lim_{x \to -1}}\sin(x+1)-\displaystyle{\lim_{x \to -1}}\ln(x+2)$. Since $\displaystyle{\lim_{x \to -1}}\sin(x+1)=\sin(0)=0$ and $\displaystyle{\lim_{x \to -1}}\ln(x+2)=\ln(1)=0$, we have $\displaystyle{\lim_{x \to -1}}\sin(x+1)-\displaystyle{\lim_{x \to -1}}\ln(x+2)=0$, and thus, $\displaystyle{\lim_{x \to -1}}(\sin(x+1)-\ln(x+2))=0$. Since $\displaystyle{\lim_{x \to -1}}(\sin(x+1)-\ln(x+2))=0$, $\displaystyle{\lim_{x \to -1}}\lvert\sin(x+1)-\ln(x+2)\rvert=0$.

\newpar
Since $0\le\lvert (\cos(\frac{1}{x+1}))(\sin(x+1)-\ln(x+2))\rvert\le\lvert \sin(x+1)-\ln(x+2)\rvert$ for all $x\neq-1, \displaystyle{\lim_{x \to -1}}\lvert\sin(x+1)-\ln(x+2)\rvert=0$ and $\displaystyle{\lim_{x \to -1}}0=0$, by the Squeeze Theorem, $\displaystyle{\lim_{x \to -1}}\lvert (\cos(\frac{1}{x+1}))(\sin(x+1)-\ln(x+2))\rvert=0$. Thus, $\displaystyle{\lim_{x \to -1}} (\cos(\frac{1}{x+1}))(\sin(x+1)-\ln(x+2))=0$. Since $\cos(\frac{1}{x+1}))(\sin(x+1)-\ln(x+2))=\sin(x+1)\cos(\frac{1}{x+1})-\ln(x+2)\cos(\frac{1}{x+1})$, we have $\displaystyle{\lim_{x \to -1}}\sin(x+1)\cos(\frac{1}{x+1})-\ln(x+2)\cos(\frac{1}{x+1})=0$, and thus $\displaystyle{\lim_{x \to -1}}f(x)=0$.

\newpar
It follows that $f(-1)=\ln(3k+2)=0$. Since $\ln(3k+2)=0$, $e^{\ln(3k+2)}=e^0$. It follows that $3k+2=1$, and therefore, $k=-\frac{1}{3}$. We have shown that $k$ must equal $-\frac{1}{3}$ for $f$ to be continuous at $x\neq-1$, completing the proof.


\end{proof}
\newpage
\noindent Jake Irons, MTH 408-01, Problem Set 3, Problem 5

\noindent ironsj@mail.gvsu.edu
\newpar
\begin{proposition}
If $f:\mathbb{R}\rightarrow\mathbb{R}$ and $\displaystyle{\lim_{x \to 1}}f(x)=L$, where $0<\lvert L\rvert<1$, then there exists a deleted
neighborhood $V_\delta^*$(1) of 1 such that $\lvert f(x)\rvert>2\lvert L\rvert-1$ for all $x\in V_\delta^*$(1).
\end{proposition}
\begin{proof}
Suppose $\displaystyle{\lim_{x \to 1}}f(x)=L$, where $0<\lvert L\rvert<1$. We want to show there exists a deleted
neighborhood $V_\delta^*$(1) of 1 such that $\lvert f(x)\rvert>2\lvert L\rvert-1$ for all $x\in V_\delta^*$(1). 

\newpar
Because $\displaystyle{\lim_{x \to 1}}f(x)=L$, there exists a deleted neighborhood $V_\delta^*$(1) of 1 such that $\lvert f(x)-L\rvert<\epsilon$ for all $x\in V_\delta^*(1)$. By Theorem R.2.1 $\lvert\lvert f(x)\rvert-\lvert L\rvert\rvert\le\lvert f(x)-L\rvert<\epsilon$ for all $x\in V_\delta^*$(1). Since $\lvert\lvert f(x)\rvert-\lvert L\rvert\rvert<\epsilon$, this means that $-\epsilon<\lvert f(x)\rvert-\lvert L\rvert<\epsilon$ for all $x\in V_\delta^*$(1). Therefore, because $-\epsilon<\lvert f(x)\rvert-\lvert L\rvert$, if we add $\lvert L\rvert$ to each side we get $\lvert L\rvert-\epsilon<\lvert f(x)\rvert$ for all $x\in V_\delta^*$(1).

\newpar
 Now, since $\lvert L\rvert<1$ is given, if we subtract $\lvert L\rvert$ from each side we get $0<1-\lvert L\rvert$. We will let $\epsilon=1-\lvert L\rvert$ since $0<1-\lvert L\rvert$. Therefore, since $\lvert L\rvert-\epsilon<\lvert f(x)\rvert$ and $\epsilon=1-\lvert L\rvert$, we get $\lvert L\rvert-(1-\lvert L\rvert)<\lvert f(x)\rvert$ for all $x\in V_\delta^*$(1). If we are to simplify the left part of the inequality the result is $2\lvert L\rvert-1<\lvert f(x)\rvert$. We have shown that if $f:\mathbb{R}\rightarrow\mathbb{R}$ and $\displaystyle{\lim_{x \to 1}}f(x)=L$, where $0<\lvert L\rvert<1$, then there exists a deleted
neighborhood $V_\delta^*$(1) of 1 such that $\lvert f(x)\rvert>2\lvert L\rvert-1$ for all $x\in V_\delta^*$(1), completing the proof.

\end{proof}

\newpar
\begin{proposition}
If $g:[0, 1]\rightarrow\mathbb{R}$ is continuous and $g(x)>0$ for all $x\in[0, 1]$, then there exists
$M>0$ such that $g(x)>M$ for all $x\in[0, 1]$.
\end{proposition}
\begin{proof}
Suppose $g:[0, 1]\rightarrow\mathbb{R}$ is continuous and $g(x)>0$ for all $x\in[0, 1]$ and we want to show there exists $M>0$ such that $g(x)>M$ for all $x\in[0, 1]$. For contradiction, we will assume there doesn't exist $M>0$ such that $g(x)>M$ for all $x\in[0,1]$. By definition C.3.1, for each $n\in\mathbb{N}$ there exists $x_n\in[0,1]$ such that $\mid g(x_n)\mid<\frac{1}{n}$. Since $g(x)>0$ for all $x\in[0,1]$, this means $0<g(x_n)<\frac{1}{n}$. Therefore, we have a sequence $(x_n)$ in [0,1] in which $0<g(x_n)<\frac{1}{n}$ for each $n\in\mathbb{N}$. Since $(x_n)\subseteq[0,1]$, $(x_n)$ must be a bounded sequence. Therefore, by The Bolzano-Weierstrass Theorem, $(x_n)$ has a convergent subsequence $(x_n_k)$. We will let $\displaystyle{\lim_{n \to \infty}}x_n_k=L$. Since $(x_n_k)\subseteq[0,1]$ and [0,1] is a closed subset of $\mathbb{R}$, $L\in[0,1]$.

\newpar
Since we have a sequence $(x_n)$ in [0,1] in which $0<g(x_n)<\frac{1}{n}$ for each $n\in\mathbb{N}$ and a convergent subsequence, $(x_{nk})$, of $(x_n)$, we have $0<g(x_n_k)<\frac{1}{n_k}$ for each $n\in\mathbb{N}$. If we are to take the limit as $n$ approaches infinity for each part of the inequality we have $\lim(0)<\lim(g(x_n_k))<\lim(\frac{1}{n_k})$. The limit of 0 is 0, and $\lim(\frac{1}{n_k})=0$. Thus, by the Squeeze Theorem, $\lim(g(x_n_k))=0$. Since it is given that $g$ is continuous on [0,1], we may say $g(\lim(x_n_k))=0$. Therefore, since $\displaystyle{\lim_{n \to \infty}}x_n_k=L$, $g(L)=0$. However, this contradicts that $g(x)>0$ for all $x\in[0,1]$. Thus, there exists $M>0$ such that $f(x)>M$ for all $x\in[0,1]$, completing the proof.



\end{proof}
\newpage
\noindent Jake Irons, MTH 408-01, Problem Set 3, Problem 7

\noindent ironsj@mail.gvsu.edu
\newpar
\begin{proposition}
There is a function $f$ such that $f$ is continuous nowhere, but $\lvert f\rvert$ is continuous everywhere.

\end{proposition}
\begin{proof}
Define $f$ as function where $f(x)=1$ if $x\in\mathbb{Q}$ and $f(x)=-1$ if $x\not\in\mathbb{Q}$. We claim that $f$ is continuous nowhere, but $\lvert f\rvert$ is continuous everywhere. We will first show that $f(x)$ is continuous nowhere. Let $c\in\mathbb{R}$. Then $c\in\mathbb{Q}$ or $c\not\in\mathbb{Q}$. We will first consider when $c\in\mathbb{Q}$. If $c\in\mathbb{Q}$, then $f(c)=1$. By the Density Property, for each $n\in\mathbb{N}$, there exists $x_n\in(c,c+\frac{1}{n})$ such that $x_n\not\in\mathbb{Q}$. Thus, $c<x_n<c+\frac{1}{n}$ for all $n\in\mathbb{N}$. Since $\lim(\frac{1}{n})=0$ and $\lim(c)=c$, we have $\lim(c+\frac{1}{n})=c$. Thus, by the Squeeze Theorem, $\lim(x_n)=c$. But, $f(x_n)=-1$ for all $n\in\mathbb{N}$ since $x_n\not\in\mathbb{Q}$. Thus, $\lim(f(x_n))=-1\neq1$. Therefore, by discontinuity criterion, $f$ is not continuous when $c\in\mathbb{Q}$.

\newpar
We now consider when $c\not\in\mathbb{Q}$. If $c\not\in\mathbb{Q}$, then $f(c)=-1$. By the Density Property, for each $n\in\mathbb{N}$, there exists $y_n\in(c,c+\frac{1}{n})$ such that $y_n\in\mathbb{Q}$. Thus, $c<y_n<c+\frac{1}{n}$ for all $n\in\mathbb{N}$. Since $\lim(\frac{1}{n})=0$ and $\lim(c)=c$, we have $\lim(c+\frac{1}{n})=c$. Thus, by the Squeeze Theorem, $\lim(y_n)=c$. But, $f(y_n)=-1$ for all $n\in\mathbb{N}$ since $y_n\in\mathbb{Q}$. Thus, $\lim(f(y_n))=1\neq-1$. Therefore, by discontinuity criterion, $f$ is not continuous when $c\not\in\mathbb{Q}$. Since $f$ is discontinuous when $c\in\mathbb{Q}$ and $c\not\in\mathbb{Q}$, $f$ is continuous nowhere.

\newpar
Since $f(x)$ is equal to 1 or -1, $\lvert f(x) \rvert=1$ for all $x\in\mathbb{R}$, making it continuous everywhere. Thus, we have shown a function $f$ that is continuous nowhere, but $\mid f\mid$ is continuous everywhere, completing the proof.



\end{proof}

\newpar
\begin{proposition}
For each real number $a$, there exist a function which is continuous at $x=a$, but not at any other points.

\end{proposition}
\begin{proof}
Let $a\in\mathbb{R}$ be arbitrary. Define $g(x)=x$ if $x\in\mathbb{Q}$ and $g(x)=a$ if $x\not\in\mathbb{Q}$ where $a\in\mathbb{R}$. We claim that for each real number $a$, $g(x)$ is continuous at $x=a$, but not any other points. We will first show that our function is not continuous at any point $x\neq a$. Let $c$ be any real number not equal to a. Then $c$ is rational or irrational. If $c$ is rational, $g(c)=c$. By the Density Property, for each $n\in\mathbb{N}$, there exists $x_n\in(c,c+\frac{1}{n})$ such that $x_n\not\in\mathbb{Q}$. Thus, $c<x_n<c+\frac{1}{n}$ for all $n\in\mathbb{N}$. Since $\lim(\frac{1}{n})=0$ and $\lim(c)=c$, we have $\lim(c+\frac{1}{n})=c$. Thus, by the Squeeze Theorem, $\lim(x_n)=c$. But, $g(x_n)=a$ for all $n\in\mathbb{N}$ since $x_n\not\in\mathbb{Q}$. Thus, $\lim(g(x_n))=a\neq c$. Therefore, by discontinuity criterion, $g$ is not continuous when $c\in\mathbb{Q}$.

\newpar
We now consider when $c\not\in\mathbb{Q}$. If $c\not\in\mathbb{Q}$, then $g(c)=a$. By the Density Property, for each $n\in\mathbb{N}$, there exists $y_n\in(c,c+\frac{1}{n})$ such that $y_n\in\mathbb{Q}$. Thus, $c<y_n<c+\frac{1}{n}$ for all $n\in\mathbb{N}$. Since $\lim(\frac{1}{n})=0$ and $\lim(c)=c$, we have $\lim(c+\frac{1}{n})=c$. Thus, by the Squeeze Theorem, $\lim(y_n)=c$. But, $g(y_n)=y_n$ for all $n\in\mathbb{N}$ since $y_n\in\mathbb{Q}$. Thus, $\lim(g(y_n))=y_n\neq a$. Therefore, by discontinuity criterion, $g$ is not continuous when $c\not\in\mathbb{Q}$. Since $g$ is discontinuous when $c\in\mathbb{Q}$ and $c\not\in\mathbb{Q}$, we have shown $g$ is not continuous when $x\neq a$.

\newpar
We will now show that $g$ is continuous at $x=a$. For all $\epsilon>0$, if there exists $\delta>0$ such that $\lvert x-a\rvert<\delta$, then $\lvert f(x)-f(a)\rvert<\epsilon$. If $a$ is rational, $g(a)=a$. If $a$ is irrational, $g(a)=a$. Hence it follows that $g(a)=a$ for all $a\in\mathbb{R}$. Therefore, $\lvert f(x)-f(a)\rvert=\lvert f(x)-a\rvert$. Thus, we have two cases. If $x\in\mathbb{Q}$, $\lvert f(x)-f(a)\rvert=\lvert x-a\rvert$. If $x\not\in\mathbb{Q}$, $\lvert f(x)-f(a)\rvert=\lvert a-a\rvert=0$. 

\newpar
Let $\epsilon>0$ be given. We choose $\epsilon=\delta$. Then for $\lvert x-a\rvert<\delta$, since $\epsilon>0$, we have $\lvert f(x)-f(a)\rvert=\lvert f(x)-a\rvert=\lvert a-a\rvert=0<\epsilon$ when $x\not\in\mathbb{Q}$. For $\lvert x-a\rvert<\delta$, we have $\lvert f(x)-f(a)\rvert=\lvert f(x)-a\rvert=\lvert x-a\rvert<\delta=\epsilon$ when $x\in\mathbb{Q}$. Thus, by the definition of continuous, $g$ is continuous at $x=a$ for any real number $a$.

\newpar
We have shown that for each real number $a$, there exists a function that is only continuous at $x=a$, completing the proof.
\end{proof}
\newpage
\noindent Jake Irons, MTH 408-01, Problem Set 3, Problem 8

\noindent ironsj@mail.gvsu.edu
\newpar
\begin{proposition}
Suppose that $f$ is continuous on $[-1, 1]$ and $\mid f(x)\mid\le1$ for every $x\in[-1, 1]$. Suppose
that $g$ is continuous on $[-1, 1]$ with $g(-1)=-1$ and $g(1)=1$. Then there exists an $x_o\in[-1, 1]$ with $f(x_o)=g(x_o)$.
\end{proposition}
\begin{proof}
If $f(-1)=-1$ and $f(1)=1$ then it would hold that $g(x_o)=f(x_o)$ if $x_o=1$ or $x_o=-1$.

\newpar
Hence we will let $f(-1)\neq-1$ and $f(1)\neq1$. Then, $f(-1)>-1$ and $f(1)<1$. Now let us consider $h:[-1,1]\rightarrow\mathbb{R}$ such that $h(x)=g(x)-f(x)$ for all $x\in[-1,1]$. Since $h$ is the sum of two continuous functions, $h$ is also continuous. Then $h(-1)=g(-1)-f(-1)=-1-f(-1)<-1-1=0$. Also, $h(1)=g(1)-f(1)=1-f(1)>1-1=0$. Therefore, by the Intermediate Value Theorem, there exists $x_o\in[-1,1]$ such that $h(x_o)=0$. Since there exists $x_o\in[-1,1]$ such that $h(x_o)=0$ and $h(x)=g(x)-f(x)$, there exists $x_o\in[-1,1]$ such that $g(x_o)-f(x_o)=0$. Hence $g(x_o)=f(x_o)$, completing the proof.


\end{proof}

\newpar
\begin{proposition}
For any real number $b$ the polynomial $p(x)=x^3+2x+b$ has a real root.

\end{proposition}
\begin{proof}
Assume that $b$ is any real number. We will show that $p(x)=x^3+2x+b$ has a real root. Since $b$ is any real number, we have three cases: $b>0$, $b<0$, or $b=0$.
Let us first consider when $b>0$. Since $p(x)=x^3+2x+b$, $p(0)=b$. Since $b>0$, $p(0)>0$. Now consider $p(-b)=(-b)^3+2(-b)+b=-b^3-b$. Since $b>0$, we have $-b^3-b<0$. Since $p(x)$ is a continuous function, the Intermediate Value Theorem tells us that $p$ will take any value between $p(-b)$ and $p(0)$ at some point in the interval [-b,0]. Since $p(0)>0$ and $p(-b)<0$, there must be a real root on the interval. Thus $p(x)$ has a real root when $b>0$.

\newpar
Now let us consider when $b<0$. Since $p(x)=x^3+2x+b$, $p(0)=b$. Since $b<0$, $p(0)<0$. Now consider $p(-b)=(-b)^3+2(-b)+b=-b^3-b$. Since $b<0$, we have $-b^3-b>0$. Since $p(x)$ is a continuous function, the Intermediate Value Theorem tells us that $p$ will take any value between $p(0)$ and $p(-b)$ at some point in the interval [0,-b]. Since $p(0)<0$ and $p(-b)>0$, there must be a real root on the interval. Thus $p(x)$ has a real root when $b<0$.

\newpar
Finally, we consider when $b=0$. If $x=0$, then $p(0)=0$. Thus, $x=0$ is a root of $p(x)$ when $b=0$. We have shown that for any real number $b$, $p(x)$ has a real root, completing the proof.


\end{proof}
\end{document}
